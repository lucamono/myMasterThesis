\chapter{Conclusions}\label{ch:conclusions}
In this work we have presented a new way to solve the grasping problem of unknown objects by employing a supervised, data driven learning approach. We collected training data by means of a try-and-test
procedure by using a 6 d.o.f. robotic manipulator with a vacuum gripper. We proposed to employ a CNN that directly learn the grasping function from RGB-D images by exploiting the image's features as a
regression problem, in particular we splitted the image into a cells grid to achive the estimation of a grasping position for each cell of the grid. We have seen that the network has learned to predict
which gripping points are acceptable for each cell or not, depending on the value of the confidence score predicted.
\newpage

\begin{acknowledgements}
siamo in finale

\emph{Luca}
\end{acknowledgements}